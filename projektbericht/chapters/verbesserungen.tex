1.
Konsolenausgabe des Clients direkt in der Webansicht anzeigen lassen.
Eventuell sollte man sich davor anmelden, damit man nur seine eigenen Logfiles sehen kann.
Dadurch hat man den Vorteil, das man sich immer und von \"uberall aus seine Logfiles ansehen kann.
Der Nutzen liegt darin, dass man sich nicht immer alle Dateien herunterladen muss.
Zudem entf\"allt die Arbeit f\"ur den Professor, der die Dateien immer hochladen muss.

2.
M\"oglichkeit bei MatchPoint durch ein beendetes Spiel zu navigieren und dann Schritt f\"ur Schritt das Spiel abzugehen und die aktuelle stelle dann exportieren k\"onnen.
Der Vorteil liegt darin, das man sich bei eventuellen Absturz den Vorherigen Zustand herunterladen kann und dann wieder einlesen kann.
Damit hat man den Vorteil gewissen Situationen leichter debuggen zu k\"onnen.

3.
In MatchPoint k\"onnte man sich in einem Spiel in der Tabelle anzeigen lassen, warum man disqualifiziert wurde.
Somit hat man bei Spielen bei denen mehrere Clients disqualifiziert wurden direkt einen \"Uberblick ob es einen Zusammenhang der Disqualifikation gibt.

4.
In MatchPoint k\"onnte man in sich bei Disqualifikation in der \"Ubersicht anzeigen lassen, ob man in Phase 1 oder Phase 2 disqualifiziert wurde.
Dadurch hat man den Vorteil, dass man sofort sieht, ob es ein eventueller bekannter Fehler ist.
Zum Beispiel kann man zum Anfang noch keine Bomben gr\"o"ser Radius 0 richtig behandeln.
Sollte nun eine solche Karte ausversehen eingelesen werden kann man sofort erkennen, dass man in der Bombenphase disqualifiziert wurde und somit evtl.\ zu Beginn diese Meldung noch ignorieren.

5.
Man k\"onnte in MatchPoint Tooltips integrieren, damit man detaillierte Informationen \"uber gewissen Inhalte erf\"ahrt.
Zum Beispiel ist Anfangs der Unterschied zwischen Stones total und Points nicht direkt klar.
Durch hovern \"uber diese Fl\"ache k\"onnte man genauere Informationen erhalten und somit w\"urde f\"ur Anf\"anger die Software leichter verst\"andlich sein.

6.
Man k\"onnte sich im Editor in Fightclub anzeigen lassen, ob es sich bei der erstellten/hochgeladenen Karte um eine Faire Map handelt.
Sollte dies nicht zutreffen k\"onnte man anzeigen wieso diese nicht Fair ist und gegenbenenfalls zudem Verbesserungen einblenden, damit man die Fairness der Karte verbessern k\"onnte.
Durch dieses Feature w\"urden weniger Unfaire Karten entwickelt werden und somit auch gerechtere Matches durchgef\"uhrt werden.