Ein Turnier wird mithilfe der Software \emph{Matchpoint} gestartet und visuell angezeigt.
Hier treten die unterschiedlichen Teams gegeneinander an um Ihre St\"arken beweisen zu k\"onnen.
Die Abgaben der einzelnen Teams werden in \emph{Fightclub} eingetragen, wo man zudem Maps erstellen und andere Karten herunterladen kann.
F\"ur beide Plattformen gibt es einige Verbesserungsvorsch\"age, damit es zuk\"unftige Semester leichter haben einen hervorragenden K.I.\ Client zu entwickeln.

\subsection{Direkte Konsolenausgabe}\label{subsec:direkte-konsolenausgabe}
Es werden zu jedem Spiel alle Konsolenausgaben gespeichert und k\"onnen im Anschluss heruntergeladen werden.
Hier erh\"alt man detaillierte Informationen \"uber seinen Client und eventuelle Abst\"urze.
Es k\"onnte jedoch darauf erweitert werden, die Konsolenausgaben in Echtzeit im Browser anzuzeigen und auch im Nachhinein alles zur Verf\"ugung stehen zu lassen.
Dies k\"onnte mit einer Datenbank und einer Kommunikation mittels Socket realisiert werden.
Nutzer m\"ussten somit nicht alle Logdateien geb\"undelt herunterladen.
Es wird Ihnen dadurch die M\"oglichkeit geboten nur einzelne Dateien herunterzuladen und von \"uberall aus live auf die Ausgaben Zugriff zu erhalten.

\subsection{Export von Spielzust\"anden}\label{subsec:export-von-spielzustaenden}
Durch Matchpoint 2 ist es den Gruppen m\"oglich beendete Spiele in beide Richtungen durchzulaufen.
Ist man disqualifiziert worden, muss man diese Situation wiederherstellen und seine Software darauf testen.
Die Herstellung dieser Situation ist jedoch sehr aufwendig und k\"onnte durch einen einfachen Export des aktuellen Spielbrettzustandes inklusiver Transitionen deutlich erleichtert werden.
Der Vorteil liegt eindeutig in dem verringerten Zeitaufwand Spiele nachtr\"aglich herunterladen, wieder einlesen und debuggen zu k\"onnen.

\subsection{Profilbasiertes Matchpoint}\label{subsec:profilbasiertes-matchpoint}
Aktuell werden ununterbrochen Turniere ausgef\"uhrt um damit die G\"ute der einzelnen Clients zu erhalten.
Man k\"onnte ein profilbasiertes Matchpoint entwickeln, bei dem sich die einzelnen Gruppen anmelden k\"onnen um ein Spiel mit einer ausgew\"ahlten Karte gegen eine ausge\"ahlte Gruppe zu starten.
Hier k\"onnten dann nur den beteiligten Gruppe diese Informationen \"uber ein solches Match erteilen.
Dadurch w\"urde man \"Anderungen am Code leichter \"uberpr\"ufen, sowie eventuelle Spezialisierungen seines Clients in echten Matches testen k\"onnen .

\subsection{Integrierte Tooltips}\label{subsec:integrierte-tooltipps}
Gerade zu Beginn sind einige Elemente und deren Bedeutung von Matchpoint f\"ur den Nutzer unklar.
Durch die Integration von Tooltips w\"urden anf\"angliche Startprobleme beseitigt werden und Matchpoint damit leicher verst\"andlich werden.

\subsection{Informationen \"uber Fairness}\label{subsec:informationen-ueber-fairness}
Beim Erstellen einer Karte sollte speziell auf die Fairness dieser Map geachtet werden.
Eine willkommene Neuerung f\"ur den Mapeditor in Fightclub w\"are eine Aussage \"uber die Fairness dieser erstellen Karte zu geben.
Zudem k\"onnte man automatisch faire Startpostionen selektieren und auf die Karte setzten.
Durch diese Neuerung w\"urde die Fairness aller Karten drastisch verbessert werden, was im Interessen aller Beteiligten sein sollte.

\bigskip
\newpage