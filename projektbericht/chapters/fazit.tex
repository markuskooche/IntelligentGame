Die Vorfreude auf dieses Wahlpflichtmodul war bereits vor dem Vorlesungsstart sehr gro"s.
Vor allem der Gedanke ein eigenes Projekt von Anfang bis Ende zu planen, entwickeln, testen und lauff\"ahig spielen zu sehen trug ma"sgeblich dazu bei.

ZOCK ist ein sehr zeitintensives, jedoch auch unglaublich spannendes Modul.
Man sammelt hier in allen erdenklichen Bereichen neue Erfahrungen, wie zum Beispiel die Planung eines Softwareprojekts, verfassen eines wissenschaftlichen Berichtes, Entwicklung von Algorithmen sowie die Zusammenarbeit in einem Team.
Vor allem, wenn gravierende Fehler entstehen oder der Client unerwartet x-mal disqualifiert wird, merkt man, dass man ein Team ist und gemeinsam die Probleme angehen und beheben muss.
Hier wird nicht nur Erfahrung im Bereich Softwareentwicklung, sondern auch im Zwischenmenschlichen gesammelt.

Die einzelnen Beteiligten, haben eine so gro"se Begeisterung f\"ur diesen Kurs entwickelt, dass nicht nur die notwendigen Aufgaben erf\"ullt wurden, sondern viele weitere Ideen in dieses Projekt geflossen sind.
Es existiert aufgrund dieser Freude an ZOCK eine neue M\"oglichkeit, Spiele und Karten detaillierter zu analysieren.
Diese M\"oglichkeit entstand bez\"uglich der Entwicklung des MapAnalyzers, wodurch erreichbare Felder erkannt werden.
Der GameAnalyzer steht in Zukunft anderen Gruppen zu Verf\"ugung, damit diese die gleichen Vorteile haben, um Spiele nachtr\"aglich zu analysieren, exportieren, modifizieren und Statistiken aufzurufen.
Es gibt zudem eine intelligente Zugsortierung, die die Probleme der naiven Zugsortierung behebt und damit extreme Leistungsopimierungen bietet.

Vor allem aber am eigentlichen Entwicklungsfortschritt sieht man, dass man Projekte nur sehr schwer durchplanen kann und man des \"Ofteren ein Refactoring betreiben, neuere Erkenntnisse einarbeiten oder komplett andere Funktionalit\"aten hinzuf\"ugen muss.
Durch diese stetigen Ver\"anderungen und Anpassungen hat das Team wertvolle Erfahrungen bez\"uglich Softwareentwicklung und Softwareplanung sammeln k\"onnen.
Besonders hervorzuheben ist, dass alle vorherigen Module in diesem Kurs Verwendung finden, sei es PG1 oder PG2 um ordentlichen Code zu produzieren, oder AD um leistungsstarke Algorithmen zu verstehen und zu entwickeln.

Im Allgemeinen ist der Aufbau dieser Lehrveranstaltung sehr gut durchdacht und strukturiert aufgebaut.
Der Professor fordert sehr viel von seinen Studenten, ist daf\"ur aber auch bereit sehr viel zu opfern.
Man erh\"alt einen fundierten Umfang \"uber die Entwicklung einer k\"unstlichen Intelligenz und wie diese schrittweise verbessert wird.
Man lernt zudem mit R\"uckschl\"agen umzugehen, da man des \"Ofteren an der Spitze kratzen kann und kurze Zeit sp\"ater wieder von anderen Clients vernichtend geschlagen wird.
Auf all diese Erfahrungen m\"ochte niemand aus dem Team verzichten.
Die Erwartungen waren gro"s, welche jedoch weit \"uberstiegen wurden.
Wer sich dieses Modul und den Arbeitsaufwand zutraut, sollte es definitiv belegen.


\bigskip
\newpage